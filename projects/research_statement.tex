\documentclass{article}
\usepackage[margin=1.15in]{geometry}
\usepackage{natbib}
\usepackage{graphicx}

\title{Research Statement}
\author{Kris Sankaran}

\begin{document}
\maketitle

During my PhD, I have done research to streamline and democratize statistical
analysis of microbiome data. This has involved three complementary types of
work,
\begin{itemize}
\item Designing example workflows: There are many approaches possible at each
  step in the microbiome analysis pipeline, from raw data to scientific
  inferences, but few references for how to assemble these steps into a coherent
  workflow. Our work in \citep{Callahan2016, Fukuyama2017} offer some basic
  guideposts.
\item Developing software packages: Sometimes the same conceptually complex or
  time-consuming task appears repeatedly across collaborative projects. This has
  motivated the creation of packages to simplify these difficult steps,
  including \texttt{structSSI} \citep{sankaran2014structssi}, \texttt{mvarVis},
  \citep{mvarvis}, \texttt{treelapse} \citep{Sankaran2017}, and
  \texttt{centroidview} \citep{centroidview}.
\item Distilling relevant literature: Sometimes the barrier to effective analysis is not
  hands-on implementation of a workflow, but knowledge of which methods are
  relevant and effective. This is especially the case for more specialized
  analysis questions, and motivated our reviews in
  \citep{sankaran2017latent}.
\end{itemize}

This research direction has been driven by on-the-ground experience
collaborating with researchers interested in the basic science and clinical
consequences of variation in the human microbiome. I have tried to develop tools
and write reviews that can make these day-to-day collaborations easier. In the
process, I have found ideas from the data visualization, probabilistic modeling,
and structured regularization literatures to be particularly useful, and they
appear frequently in my contributions. Further, we have consistently ensured
that all our studies are reproducible, sharing code on github and making data
publicly available.

\section{Workflows}

The workflow papers \citep{Callahan2016, Fukuyama2017} describe techniques for
navigating different stages of microbiome analysis, for general and longitudinal
settings, respectively. \cite{Callahan2016} is a tutorial for generating and
analyzing count matrices obtained from raw 16s sequencing data associated with a
small mouse microbiome experiment. The construction of count matrices is
accomplished using \texttt{dada2} \citep{callahan2016dada2}, while the analysis
section reviews various dimensionality reduction and testing approaches suited
to microbiome data. Since the data are sparse, high-dimensional, and rich in
phylogenetic information, proper analysis requires great care. In addition to
descriptions and examples of high-level techniques, we describe several
``tricks'' that we have found useful in past collaborations, for example,
diagnostics to guide filtering and transformation of species counts.

\cite{Fukuyama2017} is similar to \cite{Callahan2016} in its focus on sharing
code for an end-to-end workflow for microbiome analysis, but is specialized to a
perturbation study design, includes metagenomic and metabolomic data, and
discusses novel biological findings. The purpose of this study is to understand
changes in the microbiome induced by a colon cleanout, which is analogous to a
flash flood in classical ecology theory, in that it clears out an ecosystem and
frees it for recolonization. Further, in order to understand the effect of this
perturbation, it is useful to have contextual information on natural variability
in the microbiome outside of these perturbations, and the study design reflects
that. Our work describes techniques for this more specialized setting, like
regularized CCA and adaptive gPCA, a method developed by a former lab member
\citep{fukuyama2017adaptive}.

\section{Software packages}

To facilitate the kinds of analysis useful in our workflow studies, we have
designed and developed several \texttt{R} packages. One of our first efforts was
\texttt{structSSI} \citep{sankaran2014structssi}, a package implementing group
and hierarchical multiple testing procedures. Group and hierarchical multiple
testing is useful in the microbiome setting because it is often the case that
species that are phylogenetically related to one another have similar responses
to environmental changes. This phylogenetic information is directly available,
and can be used to improve power when testing. However, before
\citep{sankaran2014structssi}, group and hierarchical testing schemes developed
in the statistics literature had not been applied to microbiome data. By
implementing several methods in an \texttt{R} package, writing an accessible
discussion of the inferential properties of these testing schemes, and supplying
many example tests, we made it possible for researchers in the microbiome to
make use of modern multiple testing techniques.

Another collection of packages made use of data visualization concepts to ease
pain-points in microbiome data analysis. The first of these, \texttt{mvarVis},
streamlines the creation and interpretation of multivariate analysis ordination
techniques \citep{mvarvis}. Ordinations are often used in microbiome research to
provide a low-dimensional representation of the relationship between samples and
species abundances, but to be useful, they need to be annotated by supplemental
information, such as taxonomic membership for species or treatment status for
samples. Since different ordination techniques are implemented across various
packages, each with its own interface, this means custom code needs to be
written for each type of ordination and annotation. \texttt{mvarVis} provides a
common interface to ordination methods across multiple packages, along with
shared plotting methods. Among these plotting methods are interactive
visualizations, which allow rapid cycling through different types of annotation.

The package \texttt{treelapse} addresses the problem of comparing time series
that can be arranged hierarchically \citep{Sankaran2017}. This is useful in the
microbiome context because it is often of interest to compare species abundance
trajectories across an experiment between phylogenetic groups. For example,
species may respond differently to antibiotic treatments, with some having
become resistant. \texttt{treelapse} borrows the focus-plus-context and linked
brushing ideas, common in the data visualization literature, to make it easy to
ask which parts of a tree contain trajectories of a given shape. Conversely, it
also allows querying subsets of a tree to see what trajectories match those
nodes. This interactivity greatly speeds up the process of exploring species
dynamics, which can otherwise involve printing hundreds of species trajectory
plots.

The package \texttt{centroidview} offers a different take on the subtree
visualization problem, more directly suited to the problem of comparing
centroids corresponding to subtrees in a hierarchical clustering. It provides a
few additional visual elements -- color, a heatmap, and comparison barplots --
that enrich a \texttt{treelapse}-type display for characterizing clusters at
different levels of granularity.

\section{Literature surveys}

In contrast to workflow papers and software packages, the purpose of our
literature surveys is to address limitations in exposure to and understanding of
literature relevant to specific microbiome data analysis problems, rather than
barriers in technical implementation. A first effort towards distillation of
statistics literature relevant to microbiome applications is
\citep{sankaran2017latent}, which borrows techniques from text analysis to
provide probabilistic representations of a sample-by-species abundance matrix,
which we view as analogous to a document-term matrix. While the analogy between
document-term and species abundance matrices had been observed before, the only
previous papers applying text analysis models to the microbiome failed to
visualize model output (focusing on test performance instead) or provide code
and data.

While \citep{sankaran2017latent} considers the generic microbiome dimensionality
reduction problem, \citep{sankaran2017inference, sankaran2017survey} describe
approaches to two more specialized problems that arise in microbiome analysis:
regime detection and multitable analysis. In regime detection, the question is
to identify time segments where species trajectories switch behavior, due to
antibiotic treatment or diet change, for example. Further, there is interest
here in characterizing subsets of species with similar changes in dynamics. In
our survey, we describe statistical and machine learning techniques, including
those based on regression trees, Bayesian nonparametric Hidden Markov Models,
Gaussian Processes, and changepoint detection. We discuss differences in
underlying assumptions, modeling performance, computational efficiency, and
types of output made available.

The survey in \citep{sankaran2017survey} reviews the problem of relating several
tables two one another, motivated by an application to a study comparing body
fat composition to bacterial counts, with interest in connections between the
microbiome and host metabolism. If there were only one summary of body fat --
for example, BMI -- this is simply a high-dimensional regression problem.
However, this study provides fat and muscle mass measurements at various
locations across the body, in order to provide a richer portrait of body fat
composition than simply BMI. There are many proposals for studying such
multitable data, developed across various literatures, including statistics,
ecology, and machine learning, for example. Our work provides a unified review,
along with applications to the body composition data, with the same overall
goals of comparison and distillation as \citep{sankaran2017inference}. For
example, we discuss Canonical Correlation Analysis and Partial Least Squares
(and their regularized counterparts), exponential family CCA, and multitask
learning, among other methods. Ultimately, we hope that these reviews become an
approachable and practically useful guide for researchers facing similar
problems in microbiome data analysis.

\section{Service}

In addition to microbiome-related research, I have volunteered on multiple
data-for-good projects, through Statistics for Social Good at Stanford, the San
Francisco chapter of DataKind, and a Data Science for Social Good fellowship
through the University of Chicago. I have found that skills and perspective
cultivated through microbiome research transfer surprisingly well to social good
projects. In particular, in both types of work, I aim to help experts do their
jobs better, through improved workflows, tools, and expositions on statistical
formulation. In fact, this volunteer work can be more challenging, since many of
the government institutions and nonprofits we collaborate with haven't
used data extensively, at least traditionally.

Most of these service projects fall into two categories, namely (1) developing
tools to explore otherwise inaccessible data and (2) designing systems that
match services to participants in a (somehow) optimal way. Examples of the first
type of project are the Opioid Atlas, which visualizes historical patterns of
opioid access, which reflect inequities in palliative care, the SupplyBank.org
viewer, an application describing supply and demand for welfare resources in the
San Francisco Bay Area, and IndigoVis, a application to compare student
personality surveys with academic performance.

Examples of then second type of project are a collaboration with SEDESOL, the
social development agency in Mexico, to better match government services to
populations in need, based on a multidimensional poverty index, a project with
SparkPoint to identify whether certain combinations of employment training
programs exhibited synergistic effects, and a project with Stanford Parking and
Transportation, to determine a market segmentation for encouraging more Stanford
students and workers to use public transportation or carpools in their commutes.
I think of these projects as exercises in responsible citizenship, using the
knowledge I have acquired through my education.

\section{Future Work}

I enjoy working on a blend of workflow design, literature distillation, and
package development, motivated by real scientific problems. Ultimately, my goal
is to empower data analysis practitioners in their day-to-day work. My past
studies in the microbiome have given me perspective about the most relevant
statistical questions in this field, along with pain-points associated with
answering these questions, and it would be worthwhile to develop accessible
tools to resolve them. For example, even once multitable and regime detection
models become easy to apply, interactive visualization of results could greatly
facilitate interpretation. Further, working on the microbiome has given me
exposure to general genomics concepts, and I have an interest in applying my
perspective to other types of 'omics data, especially single-cell genomic data,
which often has structure similar to the microbiome with its zero-inflation,
high-dimensionality, and heterogeneity. I believe my depth of knowledge in
statistics, visualization, and modeling will allow me to contribute valuably to
modern, data-intensive biology problems as new technologies and types of
scientific questions arise over the years.

\bibliographystyle{plainnat}
\bibliography{refs.bib}

\end{document}

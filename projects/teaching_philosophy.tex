\documentclass{article}
\usepackage[margin=1.15in]{geometry}
\usepackage{natbib}
\usepackage{graphicx}

\title{Statement of Teaching Philosophy}
\author{Kris Sankaran}

\begin{document}
\maketitle

Over the course of my PhD career, I've served as a primary instructor for one
and a teaching assistant for six statistics department courses. In addition, I
have volunteered to teach in a few scientific and community workshops. Before my
PhD, I taught several SPLASH classes (``Paradoxes in Probability'') and was a
staff tutor at the Stanford Center for Teaching and Learning.

Through these experiences, I've had exposure to a wide range of teaching
situations -- small (8 students) and large ($>$ 140 students) classes, in person
and remote instruction, freshman to PhD-level student background, on and
off-campus lectures. To adapt to these different settings, I've found it helpful
to be grounded by a few core principles,
\begin{enumerate}
  \item Most technical concepts become clearer when explained visually and with
    concrete examples -- it helps to depart from textbook explanations in
    order to develop these perspectives.
  \item Historical context, or connections to fields outside of statistics, can
    be very motivating.
  \item It's impossible to be a good teacher without being an attentive
    listener.
  \item I believe anyone can develop mathematical and data analysis skill.
\end{enumerate}

I think it's instructive to see how these values have manifested in the teaching
experience I mentioned at the outset.

I was the primary instructor for Stats 390 -- our department's statistical
consulting workshop -- during the summer of 2017. I volunteered after having
participated in the course as a consultant for 13 quarters (spanning both my
graduate and undergraduate careers). The class involved four open consulting
office hours a week, with 2 - 3 students advising researchers from across campus
on their analysis problems, and a weekly session for consultants only, where we
discussed problems from the prior week. I was lucky to have most of
administrative and student evaluation issues determined by earlier iterations of
the course. However, I did make a few modifications, mainly due to the fact that
I accepted more undergraduate students than is typical. To make up for the fact
that these students had less formal training in statistics, I developed a $\sim
40$ page consulting ``cheat sheet'' describing methods that can be used to
resolve a wide range of the questions that appear during our office hours. I
spent half of each weekly session lecturing from these notes, which feature
summaries of real consulting problems and emphasize visual explanations -- they
are linked from my webpage. I alo dropped in on most consulting sessions, which
gave the students more opportunities to reach out and get support, and which
also allowed me to check-in with whether my lectures were helping. My teaching
evaluations are attached to this statement.

In terms of teaching assistantships, my favorite courses to organize have been
Stats 60 (Introductory Statistics), taught by Mike Baiocchi, and Stats 315B
(Data Mining), taught by Jerry Friedman. I enjoyed them because (1) it was easy
to motivate course material with real-world problems, (2) many of the ideas that
students found confusing in lecture (like hypothesis testing and gradient
boosting) could be clarified during sections or office hours through drawings at
the whiteboard. My experience is that people find it easier to identify points
of confusion once they attempt to build up geometric representations of the key
quantities involved.

Aside from departmental teaching, I have participated in a few community
workshops. In late August 2017, I was a TA for the 10-day workshop, ``Strategies
and Techniques for Analyzing Microbial Population Structures'' (STAMPS), held at
the Marine Biological Laboratory in Woods Hole, MA. As a TA, I helped
troubleshoot code when students got lost during workshops, and also helped
prepare a few TA-arranged evening sessions. It was nice to spend so much time
with microbiome researchers from outside Stanford, and part of my commitment to
workflow-related papers comes from seeing so much of that type of work be useful
to practitioners at STAMPS.

I have also volunteered with Data Carpentery, which gives free data science
courses to postdocs who would otherwise be unable to enroll in formal Stanford
courses. It was exciting to teach a class where students at once had little
technical background, but also knew that course material would be directly
applicable to their day-to-day work. I also learned some interesting tricks from
the workshop, like posting sticky-notes of different colors on each laptop,
indicating whether students were running into technical or conceptual
difficulties. Separately, I was a core member of DataKind San Francisco, an
organization that promotes the use of data in the nonprofit sphere. We organized
a few DataDives (essentially hackathons, but without the competitive stress),
and it always required careful planning to ensure each project had tasks suited
to volunteers with potentially very different backgrounds. For example, we would
arrange data preparation and visualization tasks across programming languages,
and defined roles for brainstorming discussion points and follow-up ideas to
share with our nonprofit collaborators.

I'm happy to have had opportunities to work in such diverse teaching settings. I
don't think there's anything that makes me more excited than hearing (or
sharing) a good idea from statistics, and that's certainly helped me enjoy my
work. Finally, in each of these settings, I've had to adapt and grow as a
teacher, and I expect my teaching style to evolve and improve as I respond to
what I find does and does not work in novel teaching environments.

\end{document}
